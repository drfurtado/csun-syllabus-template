% Options for packages loaded elsewhere
\PassOptionsToPackage{unicode}{hyperref}
\PassOptionsToPackage{hyphens}{url}
\PassOptionsToPackage{dvipsnames,svgnames,x11names}{xcolor}
%
\documentclass[
  letterpaper,
  DIV=11,
  numbers=noendperiod]{scrartcl}

\usepackage{amsmath,amssymb}
\usepackage{iftex}
\ifPDFTeX
  \usepackage[T1]{fontenc}
  \usepackage[utf8]{inputenc}
  \usepackage{textcomp} % provide euro and other symbols
\else % if luatex or xetex
  \usepackage{unicode-math}
  \defaultfontfeatures{Scale=MatchLowercase}
  \defaultfontfeatures[\rmfamily]{Ligatures=TeX,Scale=1}
\fi
\usepackage{lmodern}
\ifPDFTeX\else  
    % xetex/luatex font selection
\fi
% Use upquote if available, for straight quotes in verbatim environments
\IfFileExists{upquote.sty}{\usepackage{upquote}}{}
\IfFileExists{microtype.sty}{% use microtype if available
  \usepackage[]{microtype}
  \UseMicrotypeSet[protrusion]{basicmath} % disable protrusion for tt fonts
}{}
\makeatletter
\@ifundefined{KOMAClassName}{% if non-KOMA class
  \IfFileExists{parskip.sty}{%
    \usepackage{parskip}
  }{% else
    \setlength{\parindent}{0pt}
    \setlength{\parskip}{6pt plus 2pt minus 1pt}}
}{% if KOMA class
  \KOMAoptions{parskip=half}}
\makeatother
\usepackage{xcolor}
\setlength{\emergencystretch}{3em} % prevent overfull lines
\setcounter{secnumdepth}{-\maxdimen} % remove section numbering
% Make \paragraph and \subparagraph free-standing
\makeatletter
\ifx\paragraph\undefined\else
  \let\oldparagraph\paragraph
  \renewcommand{\paragraph}{
    \@ifstar
      \xxxParagraphStar
      \xxxParagraphNoStar
  }
  \newcommand{\xxxParagraphStar}[1]{\oldparagraph*{#1}\mbox{}}
  \newcommand{\xxxParagraphNoStar}[1]{\oldparagraph{#1}\mbox{}}
\fi
\ifx\subparagraph\undefined\else
  \let\oldsubparagraph\subparagraph
  \renewcommand{\subparagraph}{
    \@ifstar
      \xxxSubParagraphStar
      \xxxSubParagraphNoStar
  }
  \newcommand{\xxxSubParagraphStar}[1]{\oldsubparagraph*{#1}\mbox{}}
  \newcommand{\xxxSubParagraphNoStar}[1]{\oldsubparagraph{#1}\mbox{}}
\fi
\makeatother


\providecommand{\tightlist}{%
  \setlength{\itemsep}{0pt}\setlength{\parskip}{0pt}}\usepackage{longtable,booktabs,array}
\usepackage{calc} % for calculating minipage widths
% Correct order of tables after \paragraph or \subparagraph
\usepackage{etoolbox}
\makeatletter
\patchcmd\longtable{\par}{\if@noskipsec\mbox{}\fi\par}{}{}
\makeatother
% Allow footnotes in longtable head/foot
\IfFileExists{footnotehyper.sty}{\usepackage{footnotehyper}}{\usepackage{footnote}}
\makesavenoteenv{longtable}
\usepackage{graphicx}
\makeatletter
\newsavebox\pandoc@box
\newcommand*\pandocbounded[1]{% scales image to fit in text height/width
  \sbox\pandoc@box{#1}%
  \Gscale@div\@tempa{\textheight}{\dimexpr\ht\pandoc@box+\dp\pandoc@box\relax}%
  \Gscale@div\@tempb{\linewidth}{\wd\pandoc@box}%
  \ifdim\@tempb\p@<\@tempa\p@\let\@tempa\@tempb\fi% select the smaller of both
  \ifdim\@tempa\p@<\p@\scalebox{\@tempa}{\usebox\pandoc@box}%
  \else\usebox{\pandoc@box}%
  \fi%
}
% Set default figure placement to htbp
\def\fps@figure{htbp}
\makeatother

\KOMAoption{captions}{tableheading}
\makeatletter
\@ifpackageloaded{caption}{}{\usepackage{caption}}
\AtBeginDocument{%
\ifdefined\contentsname
  \renewcommand*\contentsname{Table of contents}
\else
  \newcommand\contentsname{Table of contents}
\fi
\ifdefined\listfigurename
  \renewcommand*\listfigurename{List of Figures}
\else
  \newcommand\listfigurename{List of Figures}
\fi
\ifdefined\listtablename
  \renewcommand*\listtablename{List of Tables}
\else
  \newcommand\listtablename{List of Tables}
\fi
\ifdefined\figurename
  \renewcommand*\figurename{Figure}
\else
  \newcommand\figurename{Figure}
\fi
\ifdefined\tablename
  \renewcommand*\tablename{Table}
\else
  \newcommand\tablename{Table}
\fi
}
\@ifpackageloaded{float}{}{\usepackage{float}}
\floatstyle{ruled}
\@ifundefined{c@chapter}{\newfloat{codelisting}{h}{lop}}{\newfloat{codelisting}{h}{lop}[chapter]}
\floatname{codelisting}{Listing}
\newcommand*\listoflistings{\listof{codelisting}{List of Listings}}
\makeatother
\makeatletter
\makeatother
\makeatletter
\@ifpackageloaded{caption}{}{\usepackage{caption}}
\@ifpackageloaded{subcaption}{}{\usepackage{subcaption}}
\makeatother

\usepackage{bookmark}

\IfFileExists{xurl.sty}{\usepackage{xurl}}{} % add URL line breaks if available
\urlstyle{same} % disable monospaced font for URLs
\hypersetup{
  pdftitle={CSUN Syllabus Best Practices},
  colorlinks=true,
  linkcolor={blue},
  filecolor={Maroon},
  citecolor={Blue},
  urlcolor={Blue},
  pdfcreator={LaTeX via pandoc}}


\title{CSUN Syllabus Best Practices}
\author{}
\date{}

\begin{document}
\maketitle


\section{CSUN Syllabus Best
Practices}\label{csun-syllabus-best-practices}

An A-Z Collection from Academic First Year Experiences and Faculty
Development

Undergraduate Studies supports excellence in teaching, learning, and
community engagement for CSUN faculty, staff, and students. To that end,
CSUN's Faculty Development and Academic First Year Experiences are
collaborating to offer CSUN faculty examples of syllabus best practices.

And for the record, your syllabus is definitively yours. It's your
intellectual property. You'll have your own ways of communicating with
students. If you don't like the sample language here, don't use it. This
website won't be offended.

\subsection{Your syllabus: the basics and some sample language
A-Z}\label{your-syllabus-the-basics-and-some-sample-language-a-z}

Start by reading the brief
\href{https://catalog.csun.edu/policies/syllabus-policy/}{CSUN Syllabus
Policy} (updated 5/28/21), which sets minimum requirements for your
syllabus. Then check out the additional topics and sample wording from
the list below.

\subsection{Absences}\label{absences}

If you (as faculty) miss a class, notify your chair (or the appropriate
staff member in your department) and your students. When you can provide
advance notification, everybody appreciates it. When you can't,
everybody understands.

\subsection{Academic honesty}\label{academic-honesty}

One of the most important steps in cultivating a course environment of
academic integrity is by explicitly communicating your expectations for
honesty. Include a section on your syllabus reminding students of the
\href{https://www.csun.edu/}{CSUN Student Conduct Code}. Spell out the
consequences you plan to impose if you find students acting without
integrity. Consider stating your expectations for each major assignment
throughout the semester (e.g., exams, papers, \& group projects). Some
faculty ask that students sign contracts or pledges.

For more information (e.g., how to prevent dishonesty \& how to respond
when students do cheat) and to see sample syllabi, visit the
\href{https://www.csun.edu/}{Faculty Development Teaching Toolkit}.

\subsection{Attendance policies}\label{attendance-policies}

On your syllabus, tell your students how you plan to handle their
absences: will you take away points after a certain number of absences?
Or will you give participation points for students who are present? Put
it in writing.

CSUN has a two-paragraph policy governing attendance:

\begin{enumerate}
\def\labelenumi{\arabic{enumi}.}
\item
  \textbf{First Paragraph:}\\
  Students are expected to attend all class meetings. Students who are
  absent from the first 2 meetings of a class that meets more than once
  a week or from the first meeting of a class that meets once a week
  lose the right to remain on the class roll and must formally withdraw
  from the class, following University procedures and deadlines. Failure
  to formally withdraw from a class will result in the instructor
  assigning to the student a grade of ``WU'' (Unauthorized Withdrawal),
  which, in computing a student's GPA, counts as a grade of ``F.''
\item
  \textbf{Second Paragraph:}\\
  In a compressed term or session of fewer than 15 weeks, the rule
  applies if the first class meeting is missed. An instructor may allow
  a student to continue in the class if the student notified the
  instructor that the absence would be temporary. If no instructor was
  assigned to the course in advance, students must notify the Department
  Chair that their absence from the class will be temporary.
\end{enumerate}

CSUN also has a Policy on Missed Classes Due to Participation in
University-Approved Activities. Participation on CSUN sports teams is
definitely included here. Key points of the policy:

\begin{itemize}
\tightlist
\item
  When representing the university in official curriculum-related,
  university-approved activities requires a student to miss classes,
  faculty are expected to provide, within reason, opportunity to make up
  any work or exams that are missed.
\item
  Absence from class for official curriculum-related,
  university-approved activities does not relieve students from
  responsibility for any part of the course work required during the
  period of absence.
\end{itemize}

\subsection{Basic Needs Assistance and
Resources}\label{basic-needs-assistance-and-resources}

Many of our students face challenges with meeting basic needs: food
insecurity, housing precarity/homelessness, mental health, healthcare,
technology, and elder/childcare. Faculty are urged to consider adding
the following language to their syllabi to support student equity,
especially during the pandemic, and link to CSUN's basic needs hub.

\textbf{CSUN with A HEART}

If you are facing challenges related to food insecurity, housing
precarity/homelessness, mental health, access to technology,
eldercare/childcare, or healthcare, you can find guidance, help, and
resources from \href{https://www.csun.edu/heart}{CSUN with A HEART}.

See also: Faculty Senate Resolution
\href{http://live-csu-northridge.pantheonsite.io}{Voluntarily Adding
Student Supportive Language to Syllabi (PDF)} (February 2021)

\subsection{Canvas: CSUN's online learning management
system}\label{canvas-csuns-online-learning-management-system}

If you choose to use Canvas, take advantage of CSUN's very helpful
technology classes and resources---including unlimited access to
\href{https://www.linkedin.com/learning/}{LinkedIn Learning} (formerly
known as lynda.com) for students \textbf{AND} faculty.

A best practice: in addition to the default Announcements feature that
is part of every Canvas course, consider adding a discussion forum
(title it ``Got Questions?'') so students can ask \textbf{AND} answer
course-related questions (such as: ``What is the new date for the
midterm? I thought I wrote it down but I can't find it!'' or ``Are we
supposed to bring our drafts to class this week or next?'' etc.). Also
consider adding an informal discussion forum (title it ``Canvas Cafe'')
and invite students to communicate informally---about a movie they saw
and loved, or their intention of going to CSUN's Big Show, or the
existence of free Associated Students academic planners available during
the first week of classes while supplies last\ldots{}

\subsection{Cell phones, tablets, laptops, and other
devices}\label{cell-phones-tablets-laptops-and-other-devices}

If you have strong and/or definite feelings about students using cell
phones, tablets, laptops, and other devices during your class, state
your policy in your syllabus.

\subsection{Classmate contact
information}\label{classmate-contact-information}

Consider including two blank lines (or some blank space) on your
syllabus near the top of page one. Ask students to get basic contact
information from two classmates (such as an email address or phone
number) so they can call on these classmates for help in case they miss
a class meeting or have a simple question.

\subsection{Contact information for you as the
professor}\label{contact-information-for-you-as-the-professor}

Include your name, office location, office hours, email address, and
CSUN phone extension.

If you have a policy about your usual time frame for responding to
student emails, you might want to state it. (Help students understand
whether you will respond immediately, within 24 hours, or before the
next class meeting---whatever seems appropriate to you. Do not promise
more than you can deliver, of course.)

Inform students if you are willing to be contacted by them in other ways
outside of class (for instance via video conferencing, discussion
forums, wikis, online chat, document sharing, cloud file storage, social
media).

\subsection{Course description}\label{course-description}

Include the complete course description in your syllabus. You can find
the official course description in the
\href{https://catalog.csun.edu}{CSUN online catalog}. You will likely
have additional information, and perhaps even your own personal course
description. Include it. Help students understand what lies ahead: tell
them so they know what to expect and how to prepare.

\subsection{Course name, number, meeting times, and room; course
contents and how you'll evaluate
them.}\label{course-name-number-meeting-times-and-room-course-contents-and-how-youll-evaluate-them.}

This information belongs on your syllabus, along with course
requirements and methods of evaluation. List the major assignments and
how many points (and/or what percentage) each contributes to the total
grade. State your grading criteria.

\subsection{DRES sample statement: accommodating students with
disabilities}\label{dres-sample-statement-accommodating-students-with-disabilities}

If you have a disability and need accommodations, please register with
the Disability Resources and Educational Services (DRES) office or the
National Center on Deafness (NCOD).\\
- The DRES office is located in Bayramian Hall, room 110 and can be
reached at 818.677.2684.\\
- NCOD is located on Bertrand Street in Jeanne Chisholm Hall and can be
reached at 818.677.2611.

If you would like to discuss your need for accommodations with me,
please contact me to set up an appointment.

\subsection{General Education course goals (student learning
outcomes)}\label{general-education-course-goals-student-learning-outcomes}

\begin{itemize}
\tightlist
\item
  \href{https://catalog.csun.edu}{GE rules for students}\\
\item
  \href{https://catalog.csun.edu}{GE student learning outcomes (SLOs)}
\end{itemize}

\subsection{Grading}\label{grading}

\subsubsection{Grading Criteria}\label{grading-criteria}

All grading criteria must be included on the syllabus.

\subsubsection{Plus/Minus Grading}\label{plusminus-grading}

I will/will not use plus/minus grading in this course.\\
\emph{(You get to choose but you are required to state your choice in
the syllabus.)}

\subsubsection{Grading for Courses with Corequisite
Labs/Discussions}\label{grading-for-courses-with-corequisite-labsdiscussions}

If students are assigned the same grade for a linked lecture and
lab/activity/discussion, the grading practice must be clearly stated on
both syllabi.

\subsection{Information competence (IC)
designation}\label{information-competence-ic-designation}

If your class carries the information competence (IC) designation in the
catalog, your syllabus should include
\href{https://catalog.csun.edu}{the IC SLOs} and your course should
enable students to meet those SLOs.

\textbf{Information Competence Goal:}\\
Students will progressively develop information competence skills
throughout their undergraduate career by developing a basic
understanding of information retrieval tools and practices as well as
improving their ability to evaluate and synthesize information
ethically.

\textbf{Information Competence Student Learning Outcomes:}\\
Students will: 1. Determine the nature and extent of information needed.
2. Demonstrate effective search strategies for finding information using
a variety of sources and methods. 3. Locate, retrieve and evaluate a
variety of relevant information, including print and electronic formats.
4. Organize and synthesize information in order to communicate
effectively. 5. Explain the legal and ethical dimensions of the use of
information.

\subsection{LRC: the Learning Resource
Center}\label{lrc-the-learning-resource-center}

Students can get free help on campus at the LRC, including: - Tutoring
in multiple subjects, - One-on-one help with writing and grammar, and -
Reading and writing workshops

The LRC offers sample language you can include in your syllabus on their
Faculty Resources page at the
\href{https://www.csun.edu/undergraduate-studies/learning-resource-center/faculty-resources}{LRC
Faculty Resources}.

\subsection{Mental Health Resources}\label{mental-health-resources}

\href{https://www.csun.edu}{University Counseling Services} (UCS) is a
mental health center for students enrolled at CSUN. UCS provides a range
of high-quality mental health services including: - Initial evaluations
- Short-term counseling and psychotherapy - Wellness Workshops - Group
treatment - Psychiatric services - Crisis/urgent care services - Case
management

UCS is located in Bayramian Hall 520 and can be reached at (818)
677-2366, Option 1.

\begin{itemize}
\tightlist
\item
  \textbf{Suicide Prevention Hotline:} (800) 273-TALK (8255) or dial 988
  to access the National Suicide and Crisis Lifeline 24 hours a day.
\item
  \textbf{Crisis Text Line:} Text ``START'' to 741-741 (Free, 24/7,
  confidential).
\end{itemize}

Scan the following QR code for more information on resources available:

\begin{figure}[H]

{\centering \pandocbounded{\includegraphics[keepaspectratio]{\#.pdf}}

}

\caption{A QR code meant to be scanned via a phone that directs users to
resources for mental health services at the UCS}

\end{figure}%

\subsection{myCSUNTablet classes}\label{mycsuntablet-classes}

If you're teaching a CSUN Tablet class (in which students are required
to use iPads), make sure you're familiar with the
\href{https://www.csun.edu}{myCSUNTablet faculty information page}.

\subsection{Office hours}\label{office-hours}

Article 20 of the Unit 3 faculty contract requires that we keep office
hours. And you will, of course, include the time and location for your
hours in your syllabus.

At CSUN, there is no campus-wide policy on how many hours to offer each
week. Some academic colleges have their own policies; others do not.
Talk to your chair or the associate dean of your college about what is
required of you.

If you need to cancel your office hours on a particular day, notify the
staff in your department office. If you are teaching an online or hybrid
class (designated OF, OC, or OH in the Schedule of Classes), consult
your chair about virtual office hours. (Also see the
\href{http://www.csun.edu/senate/policies/onlinehybridcourses.pdf}{Online
and Hybrid Courses Policy}.)

\subsection{Online, hybrid, and myCSUNtablet
classes}\label{online-hybrid-and-mycsuntablet-classes}

\textbf{Online classes:} This course meets online. Before you enroll,
take CSUN's Student Online Readiness Survey to see whether your learning
preferences and technology skills are likely to help you succeed as an
online learner. Not sure if your course will be Online? Check the
``notes'' in SOLAR Class Search for more information or review the new
course designations for hybrid and online classes.

\textbf{Hybrid classes:} This course meets partly online. Before you
enroll, take CSUN's Student Online Readiness Survey to see whether your
learning preferences and technology skills are likely to help you
succeed as an online learner. Not sure if your course will be Hybrid or
Online? Check the ``notes'' in SOLAR Class Search for more information
or review the new course designations.

\textbf{myCSUNtablet classes:} This is a myCSUNtablet class. To
participate in this class, you must bring an iPad or iPad mini running
iOS 6 or higher with a minimum of 32 GB storage. If you don't already
own one, you can purchase an iPad at the Matador Bookstore or from your
preferred Apple retailer.

\subsection{Participation}\label{participation}

State your requirements or wishes for student participation. Share your
expectations for how students should engage with the course.

\subsection{Questions from students}\label{questions-from-students}

If you have questions about the course or this syllabus, ask during
office hours or by email. Contact details are provided above, and I will
try to respond within the stated time frame.

\subsection{Required texts and other
materials}\label{required-texts-and-other-materials}

You are responsible for acquiring the following required readings and
materials for this course. \emph{(I have placed a copy of the readings
on reserve in the University Library.)}

\subsection{Savings clause: ``this syllabus is subject to
change\ldots.''}\label{savings-clause-this-syllabus-is-subject-to-change.}

This syllabus is subject to change. I will make every effort to notify
you in advance about any changes.

\subsection{Service learning classes}\label{service-learning-classes}

CSUN's Office of Community Engagement has developed a thorough set of
\href{http://live-csu-northridge.pantheonsite.io}{Helpful Hints for
Creating a Service Learning Syllabus (PDF)}. The document recommends,
among other things, that you ``create and distribute a syllabus that
clearly explains or defines the service learning goals, objectives,
criteria, and requirements'' and that you ``include the official campus
definition of Service Learning'' (which is part of the Helpful Hints
document).

\subsection{Title IX: Sexual Misconduct Disclosures/Maintaining a
Respectful Learning
Environment}\label{title-ix-sexual-misconduct-disclosuresmaintaining-a-respectful-learning-environment}

CSUN's Office of Equity and Compliance has provided
\href{http://live-csu-northridge.pantheonsite.io}{sample statements you
can use (PDF)} to help students understand how our campus handles Title
IX issues and the role that the Office of Equity and Compliance expects
faculty to play.\\
\emph{Equity and Compliance recognizes that the statements are rather
lengthy -- faculty can adopt and customize the language in ways suitable
to their teaching styles, but the core information should remain,
including the responsibility of faculty to share such info with the
Title IX Coordinator, and that students can seek confidential support
with our Campus Care Advocate.}

\subsubsection{Syllabus Resources}\label{syllabus-resources}

\begin{itemize}
\tightlist
\item
  \href{http://live-csu-northridge.pantheonsite.io}{Syllabus Rubric with
  CSUN Policy Checklist (choose PDF or DOCX)}
\item
  \href{https://www.csun.edu}{Syllabus Toolkit from Faculty Development}
\item
  \href{http://live-csu-northridge.pantheonsite.io}{Helpful Hints for
  Creating a Service Learning Syllabus (PDF)}
\item
  \href{https://catalog.csun.edu}{Catalog}
\item
  \href{https://catalog.csun.edu}{CSUN Syllabus Policy}
\item
  \href{http://live-csu-northridge.pantheonsite.io}{CSUN Syllabus
  Policy: what changed in 2015? (PDF)}
\item
  \href{https://www.csun.edu}{DRES: Disability Resources}
\item
  \href{https://www.csun.edu}{Faculty Affairs Memos}
\item
  \href{https://www.csun.edu}{Faculty Development: Syllabus Ideas and
  Policies}
\item
  \href{https://www.csun.edu}{Grievance and Grade Appeal Procedures}
\item
  \href{http://live-csu-northridge.pantheonsite.io}{Grievance and Grade
  Appeal: Faculty Reply Form}
\item
  \href{http://live-csu-northridge.pantheonsite.io}{Guide to Creating an
  Accessible Syllabus (DOCX)}
\item
  \href{https://www.csun.edu}{myCSUNtablet: Information for Faculty}
\item
  \href{http://live-csu-northridge.pantheonsite.io}{Recommended Breaks
  in Classes}
\item
  \href{https://www.csun.edu}{Senate Policies and Documents Index}
\end{itemize}

\begin{figure}[H]

{\centering \pandocbounded{\includegraphics[keepaspectratio]{\#.pdf}}

}

\caption{Syllabus icon}

\end{figure}%

\paragraph{Syllabus Articles}\label{syllabus-articles}

\begin{itemize}
\tightlist
\item
  \href{https://www.chronicle.com}{Yes, Your Syllabus Is Way Too Long}
\item
  \href{https://www.facultyfocus.com}{Seven Ways to Make Your Syllabus
  More Relevant}
\item
  \href{https://www.insidehighered.com}{Who Decides What Must Be on a
  Syllabus?}
\end{itemize}

Source:
\href{https://w2.csun.edu/educational-policies-committee/resources/syllabus-best-practices}{An
A-Z Collection from Academic First Year Experiences and Faculty
Development}




\end{document}
